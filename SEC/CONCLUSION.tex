\section{Conclusions}
In this work, the following contributions are made.
\begin{enumerate}
    \item The evolution rules for back stress and elastic core are revised.
    We ditched the idea of directly defining the back stress $\bbeta$ and the elastic core $\bc$ as independent historic variables.
    Instead, we adopt the `pseudo-polar coordinate' decomposition as shown in \eqsref{eq:decomposition_core} and \eqsref{eq:decomposition_back}.
    As a result, the proposed rules are equivalent to the original proposal while remain in simple forms that are easy to understand and implement.
    \item A simpler, universal loading/unloading condition with an algebraic interpretation is proposed.
    The procedure does not involve complex criteria and is cheap to evaluate.
    \item The specific form of evolution function $U\left(z\right)$ is further discussed.
          For numerical robustness and stability, functions that do not meet the `strong' numerical continuity condition are recommended.
    \item A reference efficient implementation is presented.
    Compared to previous implementations, the proposed implementation is more compact and efficient.
    The consistent tangent stiffness can also be explicitly evaluated.
\end{enumerate}
To summarise, the proposed formulation offers a functionally equivalent but more concise and compact alternative to the original extended subloading surface model.
This allows for a more straightforward and efficient numerical implementation without compromising accuracy and robustness.

Due to the length limitation, some additional `add-on' features, including, for example, the elastic threshold of the normal yield ratio $z$, cyclic isotropic hardening stagnation and tangential component, are not discussed in this work.
However, extending the proposed model to include these features is straightforward.
Furthermore, the proposed formulation follows the von Mises framework, thus it is suitable for metals.
It needs to be adapted for other materials.
However, the decomposition \eqsref{eq:decomposition_core} is universal and applicable to various models.

Both the uniaxial and 3D version of the proposed subloading surface model suitable for metals have been implemented in \texttt{suanPan} \citep{Chang2024}.
All numerical models and the relevant scripts are available in this repository\footnote{https://github.com/TLCFEM/subloading-implementation}.