\section{Introduction}
The conventional plasticity theory adopts the concept of yield/loading surface \citep[see, e.g.,][]{Lemaitre1990,Simo1998} to describe the onset of plastic deformation.
The surface encloses a region where the material is elastic, and once the stress state reaches the surface, plastic deformation occurs.
Because of this clear distinction between elastic and plastic states, the corresponding stiffness is often discontinuous when transiting from elastic to plastic states.
Although such a discontinuity is not physically sound, this is often not a concern for most models as the capability of capturing more accurate cyclic responses (and/or other objectives) is often more important, especially for metals.
To this end, models employ advanced/modified kinematic hardening rules enriched with various types of rate control and hardening activation/deactivation, such as Ohno's modifications \citep[see, e.g.,][]{Ohno1982,Ohno1993,Ohno2021} and Yoshida's modification \citep{Yoshida2002}, based on the famous Armstrong--Frederick rule \citep{Frederick2007} and its multiplicative version, see a comprehensive review by \citet{Chaboche2008}. have been proposed.
However, models of this kind often present decent cyclic responses under two--sided cyclic loading but are \emph{`incapable of describing the \textnormal{(general, including one--sided and small cycles)} cyclic loading behaviour'} \citep[][pg. 282]{Hashiguchi2023} due to the existence of the elastic region, regardless of large or small \citep{Dafalias1977}.

On the contrary, the subloading surface model offers a quite unique approach that is different from the conventional framework.
The initial proposal of the subloading surface model can be dated back to the work by \citet{Hashiguchi1980}.
The model is later extended \citep{Hashiguchi1989} to include an elastic core, which enables the model to describe more versatile cyclic responses.
Over the decades of development, it has been widely adopted in various applications, to name a few recent endeavours, see the work by \citet{Maranha2016,Xiong2018,Anjiki2021,Yamakawa2021,Yamakawa2022,Lu2023}.
The comprehensive review of the subloading surface model and its extended version can be found in the recent review papers by \citet{Hashiguchi2023a} and \citet{Hashiguchi2024}.
Meanwhile, the classification and overview of plasticity models is summarised and well documented in the monograph by \citet[see][\S~10.2]{Hashiguchi2023}.
Interested readers are referred to these references for more detailed information.

The (extended) subloading surface model possesses many advantages over conventional plasticity models.
The most notable one is that it provides a smooth transition between elastic and plastic states, implying the tangent stiffness is continuous under all loading conditions.
Such a continuity is more realistic and tends to be beneficial not only for static analyses, but also for other applications such as damping formulation, for example, a stiffness proportional damping based on the tangent stiffness matrix would now result in a continuous damping matrix, thus, a continuous damping force.
Furthermore, due to the introduction of the elastic core in the extended version, it is much more capable of describing ratcheting behaviour under cyclic loading, which cannot be well captured by conventional models that incorporate a pure elastic region.

The choice of internal history variables in the original formulation bears a clear and straightforward physical implication.
However, their evolution rules appear to be somewhat cumbersome and not very intuitive.
The corresponding numerical implementation \citep[see, e.g.,][]{Fincato2017,Anjiki2019,Anjiki2021} is also less efficient and robust.

In this work, we focus on the following three tasks:
\begin{enumerate}
    \item revise the definition and the corresponding evolution rule of internal history variables to make them concise and intuitive both analytically and numerically;
    \item propose a concise and universal loading/unloading criterion;
    \item improve numerical stability by discussing and choosing more numerical friendly functions for the evolution rules.
\end{enumerate}
As a showcase, a reference algorithm is also presented which adopts the von Mises framework that is suitable for metals.

The paper is organized as follows.
The original formulation of the (extended) subloading surface model is briefly reviewed in \secref{sec:subloading}.
Based on which, a revised formulation with improved evolution rules and loading/unloading criterion is proposed in \secref{sec:revised}.
A reference implementation for von Mises materials is presented in the same section.
Numerical analyses cover the comparison between the conventional and the proposed (un)loading criterion and the accuracy analysis of the proposed algorithm.